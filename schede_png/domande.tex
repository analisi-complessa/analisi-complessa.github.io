%%% Lezione
%% Preliminari di C
%% SCHEDA F
Le funzioni seno e coseno sono limitate in campo complesso.
%% END
%% RISP
    In campo complesso non vale la limitazione che sussiste in campo reale: \begin{align*}
        -1 &\le \sin x \le  1 & x&\in \R \\
        -1 &\le \cos x \le 1 & x&\in \R 
    \end{align*}e anzi, si dimostra che sono entrambe funzioni illimitate.
%% END
%% SCHEDA F
In campo complesso vale l'uguaglianza: \[\log(z^2)=2\log(z) , \qquad  \forall z \in\C , z\neq0\]
%% END
%% RISP
    Questa uguaglianza è vera soltanto per il ramo principale del logaritmo, e quindi l'affermazione è \emph{falsa}.
%% END
%% SCHEDA F
    L'esponenziale complesso è una funzione invertibile su tutto $ \C $
%% END
%% RISP
    È falso, poiché essendo periodico non è iniettivo, e quindi non è invertibile.
%% END
%% SCHEDA M
    Il logaritmo naturale del numero $ 2+2\,i $ è \begin{itemize}
        \item[(A)] $\log_{\R} (2\sqrt{2})+i(\pi/4+2k\pi)$, $ k \in \Z $, 
        \item[(B)] $\log(8)+i(\pi/4+2k\pi)$, $ k \in \Z $, 
        \item[(C)] $\log(8)+i(2\pi/3+2k\pi)$, $ k \in \Z $, 
        \item[(D)]$\log(2\sqrt{2})+i\pi/4$, $ k \in \Z $.
    \end{itemize}
%% END
%% RISP
    La soluzione è \[
        \log_{\R} (2\sqrt{2})+i(\pi/4+2k\pi),\qquad k \in \Z
    \]
%% END
%% SCHEDA M
    L'equazione \[
        e^{z} = -1,\qquad z \in \C
    \]quante soluzioni ha? \begin{itemize}
        \item[(A)] ha una soluzione; 
        \item[(B)] non ha soluzioni; 
        \item[(C)] ha un'infinità numerabile di soluzioni; 
        \item[(D)] ha un numero finito di soluzioni.
    \end{itemize}
%% END
%% RISP
    L'equazione ha un'infinità numerabile di soluzioni, quindi la risposta giusta è $(B)$. 
%% END 

%%% Lezione
%%% Funzioni olomorfe
%% SCHEDA V
L'esponenziale complesso è una funzione periodica su $ \C $.
%% END
%% RISP
    La risposta è: \emph{vero}; infatti, l'esponenziale complesso è una funzione periodica su $ \C $, con periodo $ 2\pi \,i $.
%% END
%% SCHEDA V
    Sia $f : \C \longrightarrow \C$ olomorfa, allora la sua restrizione su $\R$, ovvero $\restriction{f}{\R} : \R \longrightarrow \R$ è una funzione derivabile.
%% END
%% RISP
    Si può considerare $f = u(x,y) + i\, v(x,y)$, la restrizione di $f$ all'asse reale diventa $f = u(x,y)$, ma se richiediamo che anche il dominio sia solo l'asse reale, dobbiamo considerare $(x,y) = (x,0)$, da ciò si ha
    $\restriction{f}{\R} = u(x,0)$. 
    
    Poiché $f$ è olomorfa, $u$ è una funzione differenziabile, quindi $u(x,0)$ è derivabile e la sua derivata è 
    $f_{| \mathbb{R}}'(x) = \lim_{h \to 0} \frac{u(x+h) - u(x,0)}{h} = \frac{\partial u}{\partial x}(x,0)$
%% END
%% SCHEDA F
Le equazioni di Cauchy-Riemann sono equivalenti a richiedere che la forma differenziale $f(z) \dif z$ sia esatta.
%% END
%% RISP
    L'affermazione è falsa, poiché si richiede solo che la forma sia chiusa.
%% END
%% SCHEDA M
    In riferimento alla dimostrazione del teorema di analiticità delle funzioni olomorfe, quale delle affermazioni non è vera?

    Si sceglie la circonferenza $\gamma_z$, di raggio $r > |z - z_0|$: \begin{enumerate}
        \item[(A)] Per avere indice di $\gamma_z$ in $Z$ pari ad 1
        \item[(B)] Per poter lavorare con una serie geometrica, essendo $ \frac{|z_0 - z|}{r} < 1$
        \item[(C)] Per consentire lo scambio della serie con l'integrale, essendo $ \frac{|z_0 - z|}{r} < 1$
        \item[(D)] Per poter applicare il teorema di Cauchy sui convessi, essendo $D(z_0 ; r)$ convesso
    \end{enumerate}
%% END
%% RISP
    La risposta corretta è (D).
%% END
%% SCHEDA F
    Se $a$ è uno zero di una funzione olomorfa, allora $\exists!\, m \in \N$, $ \exists\, g$ funzione olomorfa tali che \[
        f(z) = g(z)(z-a)^m
    \]
%% END
%% RISP
    È falso, perché $ g $ non deve annullarsi in $ a $.
%% END
%% SCHEDA F
    Se $a$ è uno zero di una funzione olomorfa, allora $\exists!\, m \in \N$, $ \exists\, g$ funzione olomorfa con $ g(a)\neq 0 $ tali che \[
        f(z) = g(z)(z-a)^m
    \]
%% END
%% RISP
    Non è detto che valga su tutto il dominio della funzione: è una proprietà locale. 
%% END
%% SCHEDA V
    Ogni funzione complessa polinomiale $f$ di grado $d$ ammette un punto $w$ per cui esiste un intero $k \le  d$ tale che \[
        f^{(j)}(w) = 0, \quad \forall j < k
    \]
%% END
%% RISP
    È vero: infatti dal teorema fondamentale dell'algebra $f$ ammette esattamente $d$ zeri. Sia, quindi, $w$ uno zero di $f$. Dal teorema degli zeri per funzioni olomorfe, sappiamo che $w$ avrà ordine $k$, dove si ha che $k$ è l'ordine della prima derivata di $f$ che non si annulla in $w$.
%% END
%% SCHEDA V
    Non esistono, in campo complesso, funzioni che siano analitiche e non olomorfe
%% END
%% RISP
    L'affermazione è vera: infatti a lezione è stato dimostrato che, se $ \Omega \subseteq \C $ è una regione (ovvero un aperto connesso), vale l'uguaglianza \[
        \mathcal{A}(\Omega) = H(\Omega)
    \]
%% END
%% SCHEDA F
    Sia $\Omega$ una regione tale che se $z \in \Omega$, allora $\bar{z} \in \Omega$, e sia $f \in \mathcal{H}(\Omega)$.
    
    Allora $f(\bar{z}) \in \mathcal{H}(\Omega)$
%% END
%% RISP
    L'affermazione è falsa: un controesempio immediato è fornito dalla funzione $ f=\Id_{\C} $, che è olomorfa, mentre $ f(\bar(z))=\bar{z} $ \emph{non è} olomorfa su $ \C $.
%% END
%% SCHEDA F
    Data una funzione $ f $ olomorfa su tutto $ \C $, allora la funzione $ \overline{f} $ (ovvero il complesso coniugato di $ f $) è olomorfa su tutto $ \C $.
%% END
%% RISP
    L'affermazione è falsa. Un controesempio è la funzione $ f =\Id_{\C}  $, che è olomorfa, mentre $ \overline{f}(z)=\bar{z} $ \emph{non è} olomorfa su $ \C $.
%% END
%% SCHEDA F
    In riferimento alla dimostrazione del teorema di Cauchy sui convessi, si poteva direttamente concludere che, essendo $f(z)\dif z$ chiusa (per le equazioni di Cauchy-Riemann), una forma chiusa su un convesso è anche esatta, e dunque l'integrale è nullo.
%% END
%% RISP
    L'affermazione è falsa, perché le derivate parziali esistono di $u, v$, ma a priori non sappiamo se siano o meno continue.
%% END
%% SCHEDA M
    La disuguaglianza $|e^{z^2}| \leq e^{|z|^2}$.
    \begin{itemize}
        \item[(A)] È sempre valida
        \item[(B)] Vale solo se ci restringiamo all'asse reale
        \item[(C)] Dipende dal valore della parte reale di $z$
        \item[(D)] Dipende dal valore della parte immaginaria di $z$
        \item[(E)] È sempre falsa
    \end{itemize}
%% END
%% RISP
    La risposta corretta è la $ (A) $.
%% END
%% SCHEDA F
    Esistono funzioni olomorfe che valgono costantemente zero sull'asse immaginario.
%% END
%% RISP
    È falso: infatti gli zeri di funzioni olomorfe sono sempre isolati, inoltre l'insieme degli zeri di una funzione olomorfa è al più numerabile.
%% END
%% SCHEDA M
    Una funzione olomorfa e illimitata in ogni intorno di un punto può appettere quel punto come discontinuità eliminabile: \begin{itemize}
        \item[(A)] sempre; 
        \item[(B)] mai; 
        \item[(C)] dipende dall'espressione della funzione
    \end{itemize}
%% END
%% RISP
    La risposta giusta è la $ (B) $, poiché se la discontinuità è eliminabile allora la funzione deve essere limitata in un intorno. 
%% END
%% SCHEDA M
    Sia $f$ una funzione olomorfa e sia $a$ un suo polo. Allora, detta $Q$ la parte principale di $f$ in $a$, e presa $\gamma$ una qualsiasi curva chiusa il cui supporto non contenga $a$, allora $\int_{\gamma} Q(z) \dif z$ è pari a \begin{itemize}
        \item[(A)] Dipende dalla parametrizzazione di $\gamma$;
        \item[(B)] $2\pi i \operatorname{Ind}_{\gamma}(a) \operatorname{Res}(f,a)$
        \item[(C)] $2\pi i \operatorname{Res}(f,a)$
        \item[(D)] $2\pi i \operatorname{Ind}_{\gamma}(a)$
    \end{itemize}
%% END
%% RISP
    La risposta corretta è la $(B)$ e si rimanda alla dimostrazione del teorema dei residui
%% END
%% SCHEDA V
    Dato $\Omega\subset\C$ connesso, e una funzione olomorfa a valori reali $f:\Omega\longrightarrow\C$, $ f $ è costante su $\Omega$. 
%% END
%% RISP
    L'affermazione è vera, e questo deriva direttamente dalle equazioni di Cauchy-Riemann. Infatti, se \[
        f(x+i\,y) = u(x,y) + i\,v(x,y),\qquad x,y \in \R
    \]ed $ f $ è a valori reali, allora $ v(x,y)\equiv 0 $, e dunque $ \partial_{x} u = \partial_{y} u \equiv 0 $, e poiché $ u $ è continua su un dominio connesso, necessariamente deve essere costante.
%% END
%% SCHEDA V
    Data una funzione $F$ olomorfa sul dominio $\Omega$ convesso con $F'$ continua su $\Omega$ allora l'integrale $\int_{\gamma}F'(z)dz$ è uguale a 0 qualsiasi sia il cammino $\gamma$
%% END
%% RISP
    L'affermazione è vera, come garantisce il teorema di Cauchy sui convessi. 
%% END
%% SCHEDA F
    Esistono funzioni olomorfe su $ \C $ limitate ma non costanti.
%% END
%% RISP
    L'affermazione è falsa: questo è garantito dal Teorema di Liouville, che afferma che se una funzione $ f: \C \longrightarrow \C $ è olomorfa su tutto $ \C $ e limitata, allora è costante.
%% END
%% SCHEDA F
    Dato $Z(f)$ l'insieme degli zeri di una funzione $f:\Omega \subseteq \C\longrightarrow\C$ olomorfa su $\Omega$ allora $Z(f)$ può avere punti di accumulazione in un aperto di $\C$
%% END
%% RISP
    L'affermazione è falsa, poiché gli zeri di una funzione olomorfa sono punti isolati. 
%% END
%% SCHEDA V
    Facendo riferimento all'affermazione precedente, quest'ultima è vera in $ \R $? 

    Ovvero, data $ f: A \subseteq \R \longrightarrow \R $, $ f \in C^{ \infty}(A) $, $ Z(f) $ luogo degli zeri di $ f $ può avere punti di accumulazione in un aperto di $ \R $?
%% END
%% RISP
    Sì, ad esempio il luogo degli zeri di \[
        f(x)= x^{2}\sin \left(\frac{1}{x}\right)
    \]ha punti di accumulazione un intorno aperto dell'origine.
%% END
% %% SCHEDA V
%     Data una funzione olomorfa $f:\Omega \subseteq \C\longrightarrow\C$ diversa dalla funzione nulla, allora per ogni zero $a$ della funzione $f$ esiste una prima derivata che non si annulla mai in $a$.
% %% END
% %% RISP
    
% %% END
%% SCHEDA M
    La funzione \[f(z)=\frac{\sin(z)\cos(z)}{z^2}\qquad \forall\, z\in\C\]in $z_0=0$ ha una singolarità di tipo:
    \begin{itemize}
        \item[(A)] eliminabile;
        \item[(B)] polo di ordine $ 1 $; 
        \item[(C)] polo di ordine $ 2 $;
        \item[(D)] polo di ordine $ 3 $;
    \end{itemize}
%% END
%% RISP
    La risposta giusta è la $ (C) $, ovvero un polo di ordine $ 2 $.
%% END

%%% Lezione
%%% Integrali
%% SCHEDA M
    Risolvere il seguente integrale \[\int_0^{2 \pi} \frac{\cos(2\theta)}{2 - \cos(\theta)}\dif \theta\]
%% END
%% RISP
    La soluzione è \[
        \frac{2 \pi}{3}(7 \sqrt{3} -12 )
    \]Per trovarla si utilizza la sostituzione $z = e^{i \theta}$, da cui si ricava
    \begin{align*}
        \cos(\theta) &= \frac{z + z^-1}{2} = \frac{z^2+1}{2z}\\
        \cos(2\theta) &= \frac{z^4+1}{2z^2}
    \end{align*}

    Si integra dunque sulla circonferenza di centro origine e raggio 1 (percorsa una volta in senso antiorario), da cui si ottiene

        $\int_0^{2 \pi} \frac{\cos(2\theta)}{2 - \cos(\theta)}\dif \theta = i \int_{|z| = 1} \frac{z^4+1}{z^2(z^2-4z+1)} \dif z = 2 \pi i \left[ \operatorname{Res}(f,0) + \operatorname{Res}(f,2-\sqrt{3}) \right]$

    Dove $\operatorname{Res}(f,0) = 4i$ e $ \operatorname{Res}(f,2-\sqrt{3}) = -\frac{7}{\sqrt{3}}i $
%% END
%% SCHEDA M
Dati due polinomi reali $P(x), Q(x)$ di grado rispettivamente $p$ e $q$. 
    
Data $C^{R}_{+}$ la semicirconferenza di raggio $R>>0$ e centro l'origine, contenuta nel semipiano $\mathcal{I}:\{z>0\}$, percorsa una volta in senso antiorario. 

Sotto quali ipotesi \emph{minimali} su $p$ e $q$ vale la seguente uguaglianza? \[
    \lim_{R\to+\infty}\int_{C^{R}_{+}}\frac{P(z)}{Q(z)}\dif z = 0
\]
\begin{itemize}
    \item[(A)] qualsiasi siano $ p $ e $ q $; 
    \item[(B)] $ q>p $
    \item[(C)] $ q>p+2 $
    \item[(D)] $ q> p+1 $
\end{itemize}
%% END
%% RISP
    La risposta corretta è $ (D) $, ovvero per $ q>p+1 $.
%% END