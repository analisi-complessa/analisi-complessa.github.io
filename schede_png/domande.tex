%%% Lezione
%% Preliminari di C
%% SCHEDA F
Le funzioni seno e coseno sono limitate in campo complesso.
%% END
%% RISP
    In campo complesso non vale la limitazione che sussiste in campo reale: \begin{align*}
        -1 &\le \sin x \le  1 & x&\in \R \\
        -1 &\le \cos x \le 1 & x&\in \R 
    \end{align*}e anzi, si dimostra che sono entrambe funzioni illimitate.
%% END
%% SCHEDA M
Si può scrivere $\sin(z)$ come: \begin{itemize}
    \item[(A)] $\frac{e^{-z}-e^z}{2}$; 
    \item[(B)] $\frac{e^{-z}-e^z}{2i}$; 
    \item[(C)] $\frac{e^{-z}-e^z}{-2}$;
    \item[(D)] $\frac{e^{-z}-e^z}{-2i}$. 
\end{itemize}
%% END
%% RISP
    La risposta esatta è (D).
%% END
%% SCHEDA F
    I logaritmi complessi di un numero reale strettamente negativo possono avere parte immaginaria nulla
%% END
%% RISP
Considerando $z=x+iy$ con $x<0$ e $y\ge 0$ per ipotesi allora: 
\[\log_{\C}(z)=\log_{\R}(\left\vert z\right\vert)+i(\text{arg}(z)+2k\pi)\]dove $k\in\mathbb{Z}$

Ricordiamo che se $x<0$ e $y\ge 0$ allora  $\operatorname{arg}(z)=\operatorname{arctan}(\frac{y}{x})+\pi$ dunque 
\[\operatorname{arg}(z)=\pi.\]Quindi $\log_{\C}(z)=\log_{\R}\left\vert z\right\vert+i(\pi+2k\pi)$ e la parte immaginaria si annulla se e solo se  $i\pi(1+2k)=0$ cioè $k=\frac{-1}{2}$ che è impossibile.
%% END
%% SCHEDA F
$\log_{\C}\colon\C\setminus\{0\} \longrightarrow\C$ è una funzione continua
%% END
%% RISP
    Non è continua su tutto $ \C $, bensì solo su \[
        \C\setminus\{z\in\C \text{ t.c. } \operatorname{Re}(z) \le 0\}
    \]
%% END
%% SCHEDA F
In campo complesso vale l'uguaglianza: \[\log(z^2)=2\log(z) , \qquad  \forall z \in\C , z\neq0\]
%% END
%% RISP
    Questa uguaglianza è vera soltanto per il ramo principale del logaritmo, e quindi l'affermazione è \emph{falsa}.
%% END
%% SCHEDA F
    Il seno complesso si annulla in punti immaginari puri non nulli
%% END
%% RISP
Infatti usando la formula di addizione si ha 
\[\sin (x+iy)=\sin(x)\cosh(y)+i\cos(x)\sinh(y)= 0+0i.\]
Poiché $\cosh(y)$ è sempre positivo dalla prima si ha: $\sin(x)=0$ e questo implica che $\cos(x)\neq0$ che a sua volta implica $\sinh(y)=0$. 

Ma quest'ultima è vera se e solo se $y=0$.
%% END
%% SCHEDA F
    L'esponenziale complesso è una funzione invertibile su tutto $ \C $
%% END
%% RISP
    È falso, poiché essendo periodico non è iniettivo, e quindi non è invertibile.
%% END
%% SCHEDA M
    Il logaritmo naturale del numero $ 2+2\,i $ è \begin{itemize}
        \item[(A)] $\log_{\R} (2\sqrt{2})+i(\pi/4+2k\pi)$, $ k \in \Z $, 
        \item[(B)] $\log_{\R}(8)+i(\pi/4+2k\pi)$, $ k \in \Z $, 
        \item[(C)] $\log_{\R}(8)+i(2\pi/3+2k\pi)$, $ k \in \Z $, 
        \item[(D)]$\log_{\R}(2\sqrt{2})+i\pi/4$, $ k \in \Z $.
    \end{itemize}
%% END
%% RISP
    La soluzione è \[
        \log_{\R} (2\sqrt{2})+i(\pi/4+2k\pi),\qquad k \in \Z
    \]
%% END
%% SCHEDA F
    La parte principale di $ \log(-i) $ è $ \frac{3}{2}\pi\,i $
%% END
%% RISP
Ricordando che l'argomento sta fra $-\pi$ e $\pi$, il risultato è $\frac{-i\,\pi}{2}$ 
%% END
%% SCHEDA M
    Dimostrare che $\displaystyle \frac{w}{2\pi i}  \in \Z$ se e solo se $\displaystyle e^w=1$   
%% END
%% RISP
    $e^w=1$ se e solo se $w=\log(1)=2ik\pi$, con $k \in \Z$, dunque equivalentemente $\frac{w}{2\pi i}  \in \Z$
%% END
%% SCHEDA M
    L'equazione \[
        e^{z} = -1,\qquad z \in \C
    \]quante soluzioni ha? \begin{itemize}
        \item[(A)] ha una soluzione; 
        \item[(B)] non ha soluzioni; 
        \item[(C)] ha un'infinità numerabile di soluzioni; 
        \item[(D)] ha un numero finito di soluzioni.
    \end{itemize}
%% END
%% RISP
    L'equazione ha un'infinità numerabile di soluzioni, quindi la risposta giusta è (B). 
%% END 
%% SCHEDA M
    Dimostrare che se $e^z=1$ allora $\cos(\frac{z}{i})=1$
%% END
%% RISP
    $z=\log_{\C} (1)=2ik\pi$, con $k \in \Z$, quindi \[
        \cos\left(\frac{z}{i}\right)=\cos\left(\frac{2ik\pi}{i}\right)=\cos\left(2k\pi\right)=1, \forall k \in \Z
    \]
%% END

%%% Lezione
%%% Funzioni olomorfe
%% SCHEDA V
L'esponenziale complesso è una funzione periodica su $ \C $.
%% END
%% RISP
    La risposta è: \emph{vero}; infatti, l'esponenziale complesso è una funzione periodica su $ \C $, con periodo $ 2\pi \,i $.
%% END
%% SCHEDA V
    Sia $f : \C \longrightarrow \C$ olomorfa, allora la sua restrizione su $\R$, ovvero $\restriction{f}{\R} : \R \longrightarrow \R$ è una funzione derivabile.
%% END
%% RISP
    Si può considerare $f = u(x,y) + i\, v(x,y)$, la restrizione di $f$ all'asse reale diventa $f = u(x,y)$, ma se richiediamo che anche il dominio sia solo l'asse reale, dobbiamo considerare $(x,y) = (x,0)$, da ciò si ha
    $\restriction{f}{\R} = u(x,0)$. 
    
    Poiché $f$ è olomorfa, $u$ è una funzione differenziabile, quindi $u(x,0)$ è derivabile e la sua derivata è 
    $f_{| \mathbb{R}}'(x) = \lim_{h \to 0} \frac{u(x+h) - u(x,0)}{h} = \frac{\partial u}{\partial x}(x,0)$
%% END
%% SCHEDA M
    Cosa afferma il teorema sulle equazioni di Cauchy - Riemann?
%% END
%% RISP
Siano $ \Omega \subseteq \C $ una regione, $ z_0 \coloneqq x_0+i\,y_0 \in \Omega $, $ f:\Omega\longrightarrow \C $ e $ u,v $ le parti reali ed immaginarie di $ f $, rispettivamente, come sopra.

Allora $ f $ è olomorfa in $ z_0 $ $ \iff $ 
\begin{enumerate}
    \item $ u,v $ sono differenziabili in $ (x_0,y_0) $
    \item $ \displaystyle \begin{cases}
        \dpd{u}{x}(x_0,y_0) = \dpd{v}{y}(x_0,y_0)\\[2ex]
        \dpd{u}{y}(x_0,y_0) = -\dpd{v}{x}(x_0,y_0)
    \end{cases} $
\end{enumerate}
%% END
%% SCHEDA F
Le equazioni di Cauchy-Riemann sono equivalenti a richiedere che la forma differenziale $f(z) \dif z$ sia esatta.
%% END
%% RISP
    L'affermazione è falsa, poiché si richiede solo che la forma sia chiusa.
%% END
%% SCHEDA V
Esiste un punto in cui $\overline{z}^2$ è olomorfa.
%% END
%% RISP
Scriviamo $\overline{z}^2$ come $(x-iy)^2=x^2-y^2-2ixy$, e osserviamo che le equazioni di Cauchy-Riemann valgono in $0$.
%% END
%% SCHEDA V
    La restrizione ad $ \R $ di $f(z)=(z-\overline{z})^2$ è olomorfa.
%% END
%% RISP
Scriviamo $f(z)$ come $[x+iy-(x-iy)]^2=-4y^2$, e osserviamo che se $ z $ è reale $y=\operatorname{Im}(z)=0$ e valgono le equazioni di Cauchy-Riemann.
%% END
%% SCHEDA M
    Cosa afferma il teorema di analicità delle funzioni olomorfe?
%% END
%% RISP
    Sia $\Omega \subseteq \C$ una regione, e sia $ f \in H(\Omega) $. 
        
    Allora $ f \in \mathcal{A}(\Omega) $. 

    In particolare, per ogni $ z_0 \in \Omega$ esiste $ r>0 $ tale che \[
        f(z)= \displaystyle \sum_{n=0}^{\infty} c_{n}(z-z_0)^{n},\quad \forall\, z \in D(z_0;r) 
    \]con \[
        c_{n}=\frac{1}{2\pi\,i} \int_{\gamma} \frac{f(\xi)}{(\xi-z_0)^{n+1}}\dif \xi ,\quad \forall\, n\ge 0  
    \]dove $\gamma$ è una qualsiasi circonferenza centrata in $ z_0 $ di raggio minore di $ r $ percorsa una volta in senso antiorario.
%% END
%% SCHEDA M
    In riferimento alla dimostrazione del teorema di analiticità delle funzioni olomorfe, quale delle affermazioni non è vera?

    Si sceglie la circonferenza $\gamma_z$, di raggio $r > |z - z_0|$: \begin{enumerate}
        \item[(A)] Per avere indice di $\gamma_z$ in $Z$ pari ad 1
        \item[(B)] Per poter lavorare con una serie geometrica, essendo $ \frac{|z_0 - z|}{r} < 1$
        \item[(C)] Per consentire lo scambio della serie con l'integrale, essendo $ \frac{|z_0 - z|}{r} < 1$
        \item[(D)] Per poter applicare il teorema di Cauchy sui convessi, essendo $D(z_0 ; r)$ convesso
    \end{enumerate}
%% END
%% RISP
    La risposta corretta è (D).
%% END
%% SCHEDA F
Se $f(z)$ ha una singolarità isolata in $z_0$ e lo sviluppo in serie di Laurent di $f$ centrato in $z_0$ ha una quantità numerabile non nulla di termini a indice negativo, allora $f$ ha una singolarità essenziale in $z_0$.
%% END
%% RISP
    Nel caso in cui la quantità numerabile di termini a indice negativo fosse finita, $z_0$ sarebbe un polo per $f$.
%% END
%% SCHEDA F
    Se $a$ è uno zero di una funzione olomorfa, allora $\exists!\, m \in \N$, $ \exists\, g$ funzione olomorfa tali che \[
        f(z) = g(z)(z-a)^m
    \]
%% END
%% RISP
    È falso, perché $ g $ non deve annullarsi in $ a $.
%% END
%% SCHEDA F
    Se $ a $ è un polo semplice di $ f: \Omega \subseteq \C \longrightarrow \C $ e se esistono due funzioni $ g $ e $ h $ olomorfe in un intorno di $ a $ e tali che \begin{itemize}
        \item $ \displaystyle f(z)=\frac{g(z)}{h(z)} $; 
        \item $ h(a)= 0 $; 
        \item $ h'(a)\neq 0 $
    \end{itemize}allora \[
        \operatorname{Res}(f,a) = \frac{g(a)}{h'(a)}
    \]
%% END
%% RISP
    L'enunciato sarebbe vero se tra le ipotesi risultasse $ g(a)\neq 0 $
%% END
%% SCHEDA F
    Se $a$ è uno zero di una funzione olomorfa, allora $\exists!\, m \in \N$, $ \exists\, g$ funzione olomorfa con $ g(a)\neq 0 $ tali che \[
        f(z) = g(z)(z-a)^m
    \]
%% END
%% RISP
    Non è detto che valga su tutto il dominio della funzione: è una proprietà locale. 
%% END
%% SCHEDA V
    Ogni funzione complessa polinomiale $f$ di grado $d$ ammette un punto $w$ per cui esiste un intero $k \le  d$ tale che \[
        f^{(j)}(w) = 0, \quad \forall j < k
    \]
%% END
%% RISP
    È vero: infatti dal teorema fondamentale dell'algebra $f$ ammette esattamente $d$ zeri. Sia, quindi, $w$ uno zero di $f$. Dal teorema degli zeri per funzioni olomorfe, sappiamo che $w$ avrà ordine $k$, dove si ha che $k$ è l'ordine della prima derivata di $f$ che non si annulla in $w$.
%% END

%% SCHEDA V
    Non esistono, in campo complesso, funzioni che siano analitiche e non olomorfe
%% END
%% RISP
    L'affermazione è vera: infatti a lezione è stato dimostrato che, se $ \Omega \subseteq \C $ è una regione (ovvero un aperto connesso), vale l'uguaglianza \[
        \mathcal{A}(\Omega) = H(\Omega)
    \]
%% END
%% SCHEDA M
    In $ \R $, $ f: A \subseteq \R \longrightarrow \R $, $ f \in C^{ \infty}(A) $ implica $ f $ analitica su $ A $? E il viceversa?
%% END
%% RISP
    Sebbene $ f $ analitica su $ A $ implichi che $ f \in C^{ \infty}$, il viceversa \emph{non vale}, ed un facile controesempio è la funzione $ C^{\infty} (\R) $, ma non analitica in $ x_0=0 $: \[
        f(x)=\begin{cases}
            e^{-1/x^{2}} & x\neq 0 \\ 
            0 & x=0
        \end{cases}
    \]
%% END
%% SCHEDA F
    Sia $\Omega$ una regione tale che se $z \in \Omega$, allora $\bar{z} \in \Omega$, e sia $f \in H(\Omega)$.
    
    Allora $f(\bar{z}) \in H(\Omega)$
%% END
%% RISP
    L'affermazione è falsa: un controesempio immediato è fornito dalla funzione $ f=\Id_{\C} $, che è olomorfa, mentre $ f(\bar(z))=\bar{z} $ \emph{non è} olomorfa su $ \C $.
%% END
%% SCHEDA F
    Data una funzione $ f $ olomorfa su tutto $ \C $, allora la funzione $ \overline{f} $ (ovvero il complesso coniugato di $ f $) è olomorfa su tutto $ \C $.
%% END
%% RISP
    L'affermazione è falsa. Un controesempio è la funzione $ f =\Id_{\C}  $, che è olomorfa, mentre $ \overline{f}(z)=\bar{z} $ \emph{non è} olomorfa su $ \C $.
%% END
%% SCHEDA F
    In riferimento alla dimostrazione del teorema di Cauchy sui convessi, si poteva direttamente concludere che, essendo $f(z)\dif z$ chiusa (per le equazioni di Cauchy-Riemann), una forma chiusa su un convesso è anche esatta, e dunque l'integrale è nullo.
%% END
%% RISP
    L'affermazione è falsa, perché le derivate parziali esistono di $u, v$, ma a priori non sappiamo se siano o meno continue.
%% END
%% SCHEDA M
    La disuguaglianza $|e^{z^2}| \leq e^{|z|^2}$.
    \begin{itemize}
        \item[(A)] È sempre valida
        \item[(B)] Vale solo se ci restringiamo all'asse reale
        \item[(C)] Dipende dal valore della parte reale di $z$
        \item[(D)] Dipende dal valore della parte immaginaria di $z$
        \item[(E)] È sempre falsa
    \end{itemize}
%% END
%% RISP
    La risposta corretta è la (A).
%% END
%% SCHEDA F
    Esistono funzioni olomorfe non identicamente nulle che valgono costantemente zero sull'asse immaginario.
%% END
%% RISP
    È falso: infatti gli zeri di funzioni olomorfe sono sempre isolati, inoltre l'insieme degli zeri di una funzione olomorfa è al più numerabile.
%% END
%% SCHEDA M
    Perché nella dimostrazione olomorfia implica analitictà consideriamo $\gamma_{z}$ e non $\gamma$?
%% END
%% RISP
    Quello che si vuole fare nei passi successivi è utilizzare il teorema dell'indice di Cauchy sui convessi: ecco perché si è dovuti andare a considerare la palla aperta $D(z_{0},r)$. Inoltre, essendo aperta, non si può trovare una $\gamma$ valida $\forall\,{z}$.
%% END
%% SCHEDA M
    Una funzione olomorfa e illimitata in ogni intorno di un punto può appettere quel punto come discontinuità eliminabile: \begin{itemize}
        \item[(A)] sempre; 
        \item[(B)] mai; 
        \item[(C)] dipende dall'espressione della funzione
    \end{itemize}
%% END
%% RISP
    La risposta giusta è la (B), poiché se la discontinuità è eliminabile allora la funzione deve essere limitata in un intorno. 
%% END
%% SCHEDA M
    Sia $f$ una funzione olomorfa e sia $a$ un suo polo. Allora, detta $Q$ la parte principale di $f$ in $a$, e presa $\gamma$ una qualsiasi curva chiusa il cui supporto non contenga $a$, allora $\int_{\gamma} Q(z) \dif z$ è pari a \begin{itemize}
        \item[(A)] Dipende dalla parametrizzazione di $\gamma$;
        \item[(B)] $2\pi i \operatorname{Ind}_{\gamma}(a) \operatorname{Res}(f,a)$
        \item[(C)] $2\pi i \operatorname{Res}(f,a)$
        \item[(D)] $2\pi i \operatorname{Ind}_{\gamma}(a)$
    \end{itemize}
%% END
%% RISP
    La risposta corretta è la (B) e si rimanda alla dimostrazione del teorema dei residui
%% END
%% SCHEDA V
    Dato $\Omega\subset\C$ connesso, e una funzione olomorfa a valori reali $f:\Omega\longrightarrow\C$, $ f $ è costante su $\Omega$. 
%% END
%% RISP
    L'affermazione è vera, e questo deriva direttamente dalle equazioni di Cauchy-Riemann. Infatti, se \[
        f(x+i\,y) = u(x,y) + i\,v(x,y),\qquad x,y \in \R
    \]ed $ f $ è a valori reali, allora $ v(x,y)\equiv 0 $, e dunque $ \partial_{x} u = \partial_{y} u \equiv 0 $, e poiché $ u $ è continua su un dominio connesso, necessariamente deve essere costante.
%% END
%% SCHEDA V
    Data una funzione $F$ olomorfa sul dominio $\Omega$ convesso con $F'$ continua su $\Omega$ allora l'integrale $\int_{\gamma}F'(z)dz$ è uguale a 0 qualsiasi sia il cammino $\gamma$
%% END
%% RISP
    L'affermazione è vera, come garantisce il teorema di Cauchy sui convessi. 
%% END
%% SCHEDA F
    Esistono funzioni olomorfe su $ \C $ limitate ma non costanti.
%% END
%% RISP
    L'affermazione è falsa: questo è garantito dal Teorema di Liouville, che afferma che se una funzione $ f: \C \longrightarrow \C $ è olomorfa su tutto $ \C $ e limitata, allora è costante.
%% END
%% SCHEDA F
    Dato $Z(f)$ l'insieme degli zeri di una funzione $f:\Omega \subseteq \C\longrightarrow\C$ olomorfa su $\Omega$ allora $Z(f)$ può avere punti di accumulazione in un aperto di $\C$
%% END
%% RISP
    L'affermazione è falsa, poiché gli zeri di una funzione olomorfa sono punti isolati. 
%% END
%% SCHEDA V
    Facendo riferimento all'affermazione precedente, quest'ultima è vera in $ \R $? 

    Ovvero, data $ f: A \subseteq \R \longrightarrow \R $, $ f \in C^{ \infty}(A) $, $ Z(f) $ luogo degli zeri di $ f $ può avere punti di accumulazione in un aperto di $ \R $?
%% END
%% RISP
    Sì, ad esempio il luogo degli zeri di \[
        f(x)= x^{2}\sin \left(\frac{1}{x}\right)
    \]ha punti di accumulazione un intorno aperto dell'origine.
%% END
%% SCHEDA V
    Data una funzione olomorfa $f:\Omega \subseteq \C\longrightarrow\C$ diversa dalla funzione nulla, allora per ogni zero $a$ della funzione $f$ esiste una prima derivata che non si annulla mai in $a$.
%% END
%% RISP
    Questo è vero: infatti, ciascuno zero $ a $ ha un ordine $ m $, e pertanto \[
        f(z) = (z-a)^{m}\, g(z),\qquad g \in H(\Omega), g(a)\neq 0.
    \]
%% END
%% SCHEDA M
    La funzione \[f(z)=\frac{\sin(z)\cos(z)}{z^2}\qquad \forall\, z\in\C\]in $z_0=0$ ha una singolarità di tipo:
    \begin{itemize}
        \item[(A)] eliminabile;
        \item[(B)] polo di ordine $ 1 $; 
        \item[(C)] polo di ordine $ 2 $;
        \item[(D)] polo di ordine $ 3 $;
    \end{itemize}
%% END
%% RISP
    La risposta giusta è la (B), ovvero un polo di ordine $ 1 $.
%% END
% %% SCHEDA F
%     Data una funzione $f$ non costante olomorfa su una regione $\Omega$ e il suo modulo, la funzione $|f|$; allora $|f|$ può avere massimi locali in punti di $\Omega$.
% %% END
% %% RISP
%     \lipsum[1]
% %% END
%% SCHEDA V
Siano $f$ e $g$ due funzioni analitiche nella regione $\Omega$, con \[f(z)=g(z),\qquad \forall \, z\in B \subset\Omega\]con $ B$ aperto e connesso. Allora se $B$ ha un punto di accumulazione in $\Omega$ allora $f(z)=g(z),  \forall z\in\Omega$
%% END
%% RISP
    È vero, infatti gli zeri di funzioni analitiche \emph{non identicamente nulle} sono isolati: segue che se gli zeri di $ f-g $ (funzione analitica) non sono isolati, allora $ f-g \equiv 0$. 

    Un esempio classico di questo è fornito dalle due funzioni \begin{align*}
        f(z) &= \sin^{2}(z) + \cos^{2}(z)\\ 
        g(z) &= 1
    \end{align*}che coincidono su $ \R \subseteq \C$, e pertanto devono coincidere su tutto $ \C $.
%% END
%% SCHEDA F
    Il seguente enunciato è il teorema dell'indice di Cauchy.

    Sia $f\in H$ con $\Omega$ convesso e sia $\gamma$ un cammino chiuso. Allora $\forall\,{z}\in\Omega$ si ha che \[
        f(z)\operatorname{Ind}_{\gamma}(z)=\frac{1}{2i\pi}\int_{\gamma}\frac{f(\xi)}{\xi-z}\dif z
    \]
%% END
%% RISP
Il teorema vale $\forall\,{z}\in\Omega$ con $z\notin\gamma^{\star}$
%% END
%% SCHEDA V
$e^{\frac{-1}{z^2}}$ ha una singolarità essenziale.
%% END
%% RISP
Scrivendo lo sviluppo in serie di Laurent centrato in 0 della funzione, a partire dallo sviluppo in serie di Taylor di $e^z$, si ottengono infiniti termini di indice negativo.
%% END
%% SCHEDA F
    $ \operatorname{Re}(z) : \C \longrightarrow \C $ è una funzione analitica
%% END
%% RISP
    Non valendo le equazioni di Cauchy-Riemann, la funzione non è olomorfa, e dunque neanche analitica.
%% END
%% SCHEDA M
    Dire se le seguenti definizioni sono corrette: \begin{itemize}
        \item $a$ è una singolarità eliminabile se e solo se $\lim_{z \to a} f(z)$ esiste finito
        \item $a$ è un polo se e solo se $\lim_{z\to a} f(z)=+\infty$
        \item $a$ è una singolarità essenziale se e solo se non esistono i seguenti limiti $\lim_{z \to a} f(z)$ e $\lim_{z \to a} \left\vert f(z)\right\vert$
    \end{itemize}
%% END
%% RISP
    La definizione di polo è errata: $ a $ è un polo se e solo se \[
        \lim_{z\to a} \left\vert f(z)\right\vert = + \infty
    \]
%% END
%% SCHEDA M
Quali sono le principali differenze tra il campo reale e quello complesso?
%% END
%% RISP
    \begin{itemize}
        \item L'esponenziale complesso è periodico; 
        \item le funzioni olomorfe sono analitiche;
        \item gli zeri di funzioni olomorfe non identicamente nulle sono isolati; 
        \item in $ \C $ vale il Teorema di Liouville.
    \end{itemize}
%% END
%% SCHEDA F
    Il teorema dei residui ha il seguente enunciato.

    Sia $\Omega$ un aperto connesso, siano $a_{1},a_{2},\dots,a_{m}$ punti in $\Omega$ con $m\in\N$ e sia $f\in H\left(\Omega\setminus\{a_{i}\}\right)$ tale che $a_{i}$ siano singolarità isolate $\forall{i}$. Allora per ogni cammino chiuso $\gamma$ tale che $a_{i}\notin\gamma^{\star}$ $\forall\,{i}$ si ha: \[\int_{\gamma}f(z)\dif z=2i\pi\sum_{k=1}^m \operatorname{Ind}_{\gamma}(a_{k})\operatorname{Res}(f,a_{k})\]
%% END
%% RISP
    L'enunciato del teorema non è completo; infatti gli $a_{i}$ devono essere poli $\forall\,{i}$.
%% END

%%% Lezione
%%% Integrali
%% SCHEDA M
    Qual è la definizione di integrale su un cammino in $ \C $?
%% END
%% RISP
Sia $ f:\Omega \subseteq \C \longrightarrow \C $ continua e sia $ \gamma:[a,b]\longrightarrow \C $ un cammino tale che $ \gamma^{\star} \subseteq \Omega $. Si definisce \[
    \int_{\gamma}f(z)\dif z \coloneqq \int_{a}^{b} f\left(\gamma(t)\right)  \,\gamma'(t)\dif t
\]
%% END
%% SCHEDA M
    Risolvere il seguente integrale \[\int_0^{2 \pi} \frac{\cos(2\theta)}{2 - \cos(\theta)}\dif \theta\]
%% END
%% RISP
    La soluzione è \[
        \frac{2 \pi}{3}(7 \sqrt{3} -12 )
    \]Per trovarla si utilizza la sostituzione $z = e^{i \theta}$, da cui si ricava
    \begin{align*}
        \cos(\theta) &= \frac{z + z^{-1}}{2} = \frac{z^2+1}{2z}\\
        \cos(2\theta) &= \frac{z^4+1}{2z^2}
    \end{align*}

    Si integra dunque sulla circonferenza di centro origine e raggio 1 (percorsa una volta in senso antiorario), da cui si ottiene

        $\int_0^{2 \pi} \frac{\cos(2\theta)}{2 - \cos(\theta)}\dif \theta = i \int_{|z| = 1} \frac{z^4+1}{z^2(z^2-4z+1)} \dif z = 2 \pi i \left[ \operatorname{Res}(f,0) + \operatorname{Res}(f,2-\sqrt{3}) \right]$

    Dove $\operatorname{Res}(f,0) = 4i$ e $ \operatorname{Res}(f,2-\sqrt{3}) = -\frac{7}{\sqrt{3}}i $
%% END
%% SCHEDA M
Dati due polinomi reali $P(x), Q(x)$ di grado rispettivamente $p$ e $q$. 
    
Data $C^{R}_{+}$ la semicirconferenza di raggio $R>>0$ e centro l'origine, contenuta nel semipiano $\mathcal{I}:\{z>0\}$, percorsa una volta in senso antiorario. 

Sotto quali ipotesi \emph{minimali} su $p$ e $q$ vale la seguente uguaglianza? \[
    \lim_{R\to+\infty}\int_{C^{R}_{+}}\frac{P(z)}{Q(z)}\dif z = 0
\]
\begin{itemize}
    \item[(A)] qualsiasi siano $ p $ e $ q $; 
    \item[(B)] $ q>p $
    \item[(C)] $ q>p+2 $
    \item[(D)] $ q> p+1 $
\end{itemize}
%% END
%% RISP
    La risposta corretta è (D), ovvero per $ q>p+1 $.
%% END
%% SCHEDA M
Qual è il risultato del seguente integrale?
\[\int_0^{2\pi} \frac{\dif\theta}{4\cos\theta+5}\]
\begin{itemize}
    \item[(A)] $ 0 $;
    \item[(B)] $\frac{2}{3} \pi $;
    \item[(C)] $\pi$; 
    \item[(D)] $\frac{\pi}{2}$.
\end{itemize}
%% END
%% RISP
    La risposta corretta è la (B).
%% END
%% SCHEDA M
    Qual è il risultato del seguente integrale?
    \[
        \int_{-\infty}^{+\infty}\frac{e^{2ix}}{1+x^2}\dif x
    \]
    \begin{itemize}
        \item[(A)] $\pi/e$;
        \item[(B)] $0$;
        \item[(C)] $\pi$; 
        \item[(D)] $i\pi/\sqrt{2}$.
    \end{itemize}
%% END
%% RISP
    La risposta corretta è la (A). 
%% END
%% SCHEDA M
    Calcolare le singolarità della seguente funzione e classificarle. \[
        f(z)=\frac{e^{\pi zi}}{(z^2+1)(z^2+25)}
    \]
%% END
%% RISP
    Le singolarità sono: 
   \[
       z_1 = i,\quad z_2=-1,\quad z_3=5i,\quad z_4=-5i
   \]Tutti i poli sono semplici, poiché per tutti esiste finito e non nullo il limite \[
    \lim_{z\to z_j} (z-z_j)f(z),\qquad j=1,2,3,4
   \]Calcoliamo ad esempio il limite per $ z_1 $: \begin{align*}
        \lim_{z\to z_1} (z-z_1)f(z) &=\lim_{z\to i} (z-i)\frac{e^{\pi z\,i}}{(z+1)(z-i)(z+5i)(z-5i)}\\ 
        &= \frac{e^{-\pi}}{48\,i}
   \end{align*}
%% END
%% SCHEDA M
    Utilizzando la risposta della domanda precedente, calcolare il seguente integrale: \[
        \int_0^{+\infty}\frac{\cos{\pi x}+2}{(x^2+1)(x^2+25)}\dif x
    \]
%% END
%% RISP
    Si ha che \[
        \int_0^{+\infty}\frac{\cos{\pi x}+2}{(x^2+1)(x^2+25)}\dif x=\pi (\frac{e^{-\pi}}{48}-\frac{e^{-5\pi}}{240})+\frac{\pi}{30}
    \]
%% END